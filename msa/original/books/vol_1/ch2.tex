\chapter{Getting Started on the $N$-Body problem}

Our goal is build a laboratory to study the interactions between stars.
Since stars don't fit in traditional laboratories, we have no choice
but to use virtual stars in virtual labs.  The computer provides us
with the right virtual environment, and it is our task to write the
software that will correctly simulate the behavior of the virtual stars
and their interactions.  Once that software is in place, or at least
enough of it to start playing, the user can provide a starting situation,
after which our software will evolve the system, for a few billion years,
say.

In this book we will focus in detail on the whole process of developing
the software needed.  We will aim at realistic detail, showing the way
of thinking that underlies the construction of a complex and ever-growing
software environment.  We will require patience from the reader, since it
will take a while to have a full package in hand for modeling, say,
the long-term behavior of a star cluster.  This drawback, we feel, is
more than offset by the advantages of our approach:

\begin{itemize}
\item
the reader will be fully empowered to customize {\it any}
aspect of the software environment or {\it any} larger or smaller part of it;
\item
the reader will be able to use the package with complete
understanding and appreciation of what are and are not reasonable ways
to apply the tools;
\item
the reader will learn to embark on completely different large-scale
software projects, be they in (astro)physics, other areas of science,
or other fields altogether; 
\item
and in addition, we hope that reading these books will be as much fun
for the reader as it was for us to write them.
\end{itemize}

\section{Our Setting}

We want to convey some of the atmosphere in which large software
environments are grown, in a dynamic and evolutionary way, often
starting from very small beginnings -- and always surprising the
original authors when they see in what unexpected ways others can
and will modify their products in whole new directions.  Most of our
narrative will follow this process step by step, but occasionally we
will turn away from the process to the players: the developers writing
the software.  We have chosen one representative of each of the three
target groups mentioned in our preface, from natural science, business
and computer science.

The setting is an undergraduate lounge, where three friends occasionally
hang out after dinner, and sometimes tell each other about the projects
they are working on.  Tonight, Alice talks with great animation to her
friends Bob and Carol.  Alice is an astrophysics major, Bob is preparing
for business school, and Carol majors in computer science.

\abc

\alice
Guess what!  Today I was asked to choose a junior project, for
me to work on for half a year.  Many of the choices offered seemed to
be interesting, but for me the most exciting opportunity was to work
on the overhaul of a laboratory for interactions between stars.

\bob
What are the interactions that are so interesting?

\alice
Imagine this, the current software package allows you to create
a star cluster and to let it evolve for billions of years, and then
you can fly through the whole four-dimensional history in space and
time to watch all the collisions and close encounters between normal
stars and black holes and white dwarfs and you name it!

\carol
If that package already exists, what then is so exciting about
an overhaul?

\alice
Yes, the package exists, but every large software package tends
to grow and to become overweight.  As you both know, this is true in
business-driven software projects, but it is even more true in science
settings, where the value of clean software engineering is underrated
even more than in more profit-oriented areas.  As a result, by far
the most reasonable and efficient way to extend older packages is first
to do a thorough overhaul.

\bob
I see.  You mean that rewriting a package is worth the time,
presumably because you have already figured out the physics and you
have similarly built up extensive experience with hooking everything
together in various ways in software.

\alice
Exactly.  Rewriting a package takes far less time than writing
it in the first place -- if we want to keep the same functionality.
In practice, it may take longer than we think, since we will for sure
find new extensions and more powerful ways to include more physics.
As long as we don't get carried away, and keep our science goals in
sight, this extra time is well spent and will lead to greater
productivity.

\carol
I wonder, though, whether a complete overhaul is desirable.  I
have just learned about a notion called {\it refactoring}.  The idea
is to continuously refine and clean up code while you go along.

\alice
Yes, that would be better.  In fact, I already had a brief chat
with my supervisor, and he mentioned just that.  He said that this was
the last really major overhaul he hoped to do for his software
environment.  The main reason for the planned overhaul is to make
it flexible enough that the system from now on can grow more
organically.

\bob
The overhaul that will be the end of all overhauls!

\carol
Well, maybe.  I've heard a lot of hype about programming, in
the few years that I have been exposed to it.  But the basic idea
sounds good.  And even if you will have to overhaul in the future, a
cleaner and more modular code will surely be easier to understand and
disentangle and hence to overhaul.

\bob
May I ask a critical question?  You have half a year to get your
feet wet, doing a real piece of scientific research.  Would it really
be prudent to spend that time overhauling someone else's code?

\alice
I asked that question, too.  My supervisor told me that a
through-going attempt to improve a large software environment in a
fundamental way from the bottom up is guaranteed to lead to new
science.  Instead of overhauling, a better term might be fermenting.
You will reap the benefits of all the years of experience that have
gone into building the software, from working with the science to the
figuring out of the architecture of the software environment.  Those
who write the original code won't have the time to do a complete
rewrite; they have become too engrossed in teaching and administration.
But they will have time to share their experience, and they will
gladly do so when they see someone seriously working on improvements.

\carol
In other words, during this coming half year you might find
yourself engaging in actual research projects, as a form of spin-off
of the overhauling, or fermenting as you just called it?

\alice
Exactly.

\bob
You know what?  Perhaps this is a silly thing to suggest, but I
suddenly got an idea.  It seems that Alice today has started what
amounts to an infinite task.  She will have her hands full at it, even
if she could clone herself into several people working simultaneously,
and she is not expected to reach anywhere near completion in half a year.
At the same time, she is expected to start absolutely from start.  If
she wouldn't do so, it wouldn't be a complete overhaul.  Here is my
proposal: how about all three of us pitching in, a couple times a
week, after dinner, using the time we tend to spend here anyway?

\carol
To keep Alice honest?

\bob
Exactly!  Of course, she may well get way ahead of us into all
kind of arcane astrophysics applications, but even so, if we plod
behind her, asking her questions about each and every decision she has
made from the start, we will probably keep her more honest than any
astrophysicist could -- simply because we know less about astrophysics
than any astrophysicist!  And besides, for me too it would be a form of
fun and profit.  I intend to focus on the software industry when I go
to business school, and I might as well get some real sense of what is
brewing in the kitchen, when people write non-trivial software systems.

\carol
Hmm, you have a point.  Obviously, something similar holds for
me too, in that I can hone my freshly learned theoretical knowledge on
realistic astrophysics problems.  What do you think, Alice, are we
rudely intruding upon your new project?

\alice
No, on the contrary!  As long as I keep my actual project
separated, as Bob stressed, I am more than happy to discuss the basics
with you both during after-dinner sessions, as long as you have the
stamina and interest to play with orbital dynamics of stars and star
systems.  And I'm sure we will all three learn from the experience: I
would be very surprised if you wouldn't inject new ideas I hadn't
thought about, or notice old ideas of mine that can be improved.

\bob
Okay, we have a deal!  Let's get started right away, and get back
here tomorrow, same time, same place.

\carol
Okay, but let's say almost the same place: next door is the
computer center, where we will be able to find a big enough screen so
that the three of us can gather around it, to start writing our first
star-moving program.

\alice
An $N$-body code, that is what it is called.  Okay, I'm game too.
See you both tomorrow!

\cba

\section{The Gravitational $N$-Body Problem}

The next day, our three friends have gathered again, ready to go.

\abc

\alice
Hi, you're all back, so I guess you were really serious.  Okay, let's
write our first code for solving the gravitational $N$-body problem.

\bob
I understand that we are dealing with something gravitational
attractions between celestial bodies, but what is the problem with that?

\carol
And why are you talking about $N$ bodies, and not $p$ bodies or
anything else?

\alice
Traditionally, in mathematics and mathematical physics, when we pose a
question, we call it a problem, as in a home work problem.  The
gravitational 2-body problem is defined as the question: given the
initial positions and velocities of two stars, together with their
masses, describe their orbits.

\bob
What if the stars collide?

\alice
For simplicity, we treat the stars as if they are mass points, without
any size.  In this case they will not collide, unless they happen to
hit each other head-on.  Of course, we can set two point masses up
such that they will hit each other, and we will have to take such
possibilities into account (see volume 2).  However, when we start
with random initial conditions, the chance of such a collision is
negligible.

\carol
But real stars are not points?

\alice
True.  At the goal of building a laboratory for star cluster evolution
is to introduce real stars with finite sizes, nuclear reactions, loss
of radiation and mass, and all that good stuff.  But we have to start
somewhere, and a convenient starting place is to treat stars as point
masses.  In practice, to discriminate between the physical modeling of
stars and the replacement of them with point masses, we often call
those points `bodies'.

This brings me to Carol's question: why do astrophysicists talk about
$N$-body simulations?  This is simply a historical convention.  I
would prefer the term many-body simulations, but somehow somewhere
someone stuck in the variable $N$ as a place-holder for how many
bodies where involved, and we seem to be stuck with that notation.

\carol
Fine.  Let's pick a language and start coding!  I bet you physics
types insist on using fortran?

\alice
Believe it or not, most of the code to be overhauled has been written
in C++, and I suggest that we adopt the same language.  It may not be
exactly my favorite, but it is at least widely available, well
supported, and likely to stay with us for decades.

\bob
What is C++, and why the obscure name?  Makes the notion of an $N$-body
seem like clarity itself!

\carol
Long story.  I don't know whether there was ever a language A, but
there certainly was a language B, which was followed alphabetically by
a newer language C, which became quite popular \dots

\bob
\dots are you making a pun on our names?

\carol
No, I'm not kidding.  Then C was extended to a new language for
object-oriented programming, something we'll talk about later.  In a
nerdy pun, the successor operation ``++'' from the C language was used
to indicate that C++ was the successor language to C.  Don't look at
me, we'll have to live with it.

\cba

\section{The Gravitational $2$-Body Problem}

A decision was made to let Carol take the controls, for now.  Taking
the keyboard in front of a large computer screen, she opens a new file
{\st nbody.C} in her favorite editor.  Expectantly, she looks at Alice,
sitting to her left, for instructions, but Bob first raises a hand.

\abc

\bob
I'm a big believer in keeping things simple.  Why not start by coding up
the 2-body problem first, before indulging in more bodies?  Also, I
seem to remember from an introductory physics class for poets that the
2-body problem was solved, whatever that means.

\alice
Good point.  Let's do that.  It is after all the simplest case that is
nontrivial: a 1-body problem would involve a single particle that is
just sitting there, or moving in a straight line with constant velocity,
since there would be no other particles to disturb its orbit.

And yes, the 2-body problem can be solved analytically.  That means
that you can write a mathematical formula for the solution.  For
higher values of $N$, whether 3 or 4 or more, no such closed formulas
are known, and we have no choice but to do numerical calculations in
order to determine the orbits.  For $N=2$, we have the luxury of being
able to test the accuracy of our numerical calculations by comparing
our results with the formula that Newton discovered for the 2-body
problem.

Yet another reason to start with $N=2$ is that the description can be
simplified.  Instead of giving the {\it absolute} positions and
velocities for each of the two particles, with respect to a given
coordinate system, it is enough to deal with the {\it relative}
positions and velocities.  Instead of dealing with position $\br_1$
for the first particle and $\br_2$ for the second particle, we can
write down the gravitational attraction between the two in terms of
the relative position, defined as:

\begin{equation}
\br = \br_2 - \br_1
\end{equation}

Newton's gravitational equation of motion then becomes:

\begin{equation}
\frac{d^2}{dt^2}\br = - G \frac{M_1 + M_2}{r^3}\br
\end{equation}

This is a second-order differential equation.  At the left-hand side
you see the second derivative of position with respect to time $t$.  The
first time derivative of position $\br$ is the velocity $\bv = d\br /dt$
while the second derivative presented here is the acceleration
$\ba = d\bv /dt = d^2\br /dt^2 $.  At the right hand side, the masses
of the two particles are indicated by $M_1$ and $M_2$, respectively.
$G$ is the value of Newton's gravitational constant.

I'm glad you both have at least some familiarity with differential
equations, in the context of classical mechanics.  It may not be a bad
idea to brush up your knowledge, if you want to know more about the
background of Newtonian gravity.  There are certainly plenty of good
introductory books.  At this point it is not necessary, though, to go
deep into all that.  I can just provide the few equations we need to
get started, and for quite a while our main challenge will be to
figure out how to solve these equations.

\bob
The differential equation does indeed look familiar, but why is there
a power 3 in the denominator?  I thought that Newtonian gravity is an
inverse square power, so I would have expected a power 2 down there.

\alice
Good question!  We are working here in three dimensions, because that
is how many dimensions space in the universe has.  The bold-face
notation $\br$ indicates that we are dealing with three-dimensional
vectors.  If we name the components as follows,

\begin{equation}
\br = \{x, y, z\}
\end{equation}

then the scalar distance between the two particles is defined by

\begin{equation}
r = | \br | = \sqrt{x^2+y^2+z^2}
\end{equation}

And while it is true that Newtonian gravity is a $1/r^2$ force, we
have to tell the particles not only the magnitude of their mutual
attraction, but also the direction in which they pull each other.
This is accomplished by adding the last factor $\br$.  To compensate
for the fact that $\br$ grows linearly with the distance, we have to
add an extra power in the denominator.

\carol
Is there no cleaner way to write this equation, making the $1/r^2$
nature of the interaction more transparent?

\alice
Sure there is.  We can introduce a so-called unit vector, which by
definition has a length of one unit in our coordinate system.  This
is a good tool for dealing with directions without introducing changes
in magnitude.  The unit vector corresponding to $\br$ is given as
$ \hat\br = \br / r $, and the equation of motion for our 2-body
problem now reads:

\begin{equation}
\frac{d^2}{dt^2}\br = - G \frac{M_1 + M_2}{r^2}\hat\br
\end{equation}

\bob
That looks more like the real thing.

\carol
Yes, but it may be easier to program the previous expression, so let's
keep both on the table for now, and see what's most convenient.

\alice
One more thing: let's make life as simple as we can, by choosing a
system of physical units in which the gravitational constant and the
total mass of the 2-body system are both unity:

\begin{eqnarray}
G & = & 1 \\
M_1 + M_2 & = & 1
\end{eqnarray}

Our original equation of motion now becomes simply:

\begin{equation}
\frac{d^2}{dt^2}\br = - \frac{\br}{r^3} \label{newton-2-bodies}
\end{equation}

\cba
