\documentstyle[epsf,11pt,verbatim]{article}

\font\lggggb=cmbx10 scaled \magstep5
\font\lgggb=cmbx10 scaled \magstep4
\font\lggb=cmbx10 scaled \magstep3
\font\lgb=cmbx10 scaled \magstep2
\font\lb=cmbx10 scaled \magstep1
\font\lgr=cmr10 scaled \magstep2
\font\lr=cmr10 scaled \magstep1
\font\slgh=cmr10 scaled \magstep2
\font\sb=cmbx9
\font\sr=cmr9
%
\renewcommand{\floatpagefraction}{0.9}
%
\def\simlt{\hbox{ \rlap{\raise 0.425ex\hbox{$<$}}\lower 0.65ex\hbox{$\sim$} }}
\def\simgt{\hbox{ \rlap{\raise 0.425ex\hbox{$>$}}\lower 0.65ex\hbox{$\sim$} }}
\def\sub#1{_{\rm #1}}
\def\sup#1{^{\rm #1}}
\def\undertext#1{{$\underline{\hbox{#1}}$}}
\def\doubleundertext#1{{$\underline{\underline{\hbox{#1}}}$}}
\def\half{{\scriptstyle {1 \over 2}}}
\def\ie{{\it {\frenchspacing i.{\thinspace}e. }}}
\def\eg{{\frenchspacing e.{\thinspace}g. }}
\def\cf{{\frenchspacing\it cf. }}
\def\etal{{\frenchspacing\it et al.}}
\def\et{{\etal}}
\def\simlt{\hbox{ \rlap{\raise 0.425ex\hbox{$<$}}\lower 0.65ex\hbox{$\sim$} }}
\def\simgt{\hbox{ \rlap{\raise 0.425ex\hbox{$>$}}\lower 0.65ex\hbox{$\sim$} }}
\def\solar{\odot}
\def\msun{\ifmmode{M_\solar}\else{$M_\solar$}\fi}
\def\rsun{\ifmmode{R_\solar}\else{$R_\solar$}\fi}
\def\Rf{\parindent=0pt\smallskip\hangindent=3pc\hangafter=1}
\def\pc{{\rm pc}}
\def\kpc{{\rm kpc}}
\def\Mpc{{\rm Mpc}}
\def\yr{{\rm yr}}
\def\Myr{{\rm Myr}}
\def\Gyr{{\rm Gyr}}
\def\kT{\ifmmode{kT}\else{$kT$}\fi}
\def\N{{\ifmmode{N}\else{$N$}\fi}}
\def\fb{{\ifmmode{f_B}\else{$f_B$}\fi}}
\def\emax{{\ifmmode{E_{max}}\else{$E_{max}$}\fi}}
\def\td{{\ifmmode{t_d}\else{$t_d$}\fi}}
\def\tcr{{\ifmmode{t_{cr}}\else{$t_{cr}$}\fi}}
\def\tr{{\ifmmode{t_r}\else{$t_r$}\fi}}
\def\trh{\ifmmode{t_{rh}}\else{$t_{rh}$}\fi}
\def\vv{{\ifmmode{\langle v^2\rangle}\else{$\langle v^2 \rangle$}\fi}}
\def\v{{\ifmmode{\langle v^2\rangle^{1/2}}
                \else{$\langle v^2 \rangle^{1/2}$}\fi}}
\def\half{{\ifmmode{{1 \over 2}}\else{${1 \over 2}$}\fi}}
\def\dhalf{{\textstyle {1 \over 2}}}
\def\threehalf{{\ifmmode{{3 \over 2}}\else{${3 \over 2}$}\fi}}
\def\dthreehalf{{\textstyle {3 \over 2}}}
\def\dfivehalf{{\textstyle {5 \over 2}}}
\def\dfivethree{{\textstyle {5 \over 3}}}
\def\kms{\ifmmode{\rm km\,s^{-1}}\else{$\rm km\,s^{-1}$}\fi}
\def\kmps{{\rm km/s}}
\def\piet#1{{\bf[#1 -- piet]}}
\def\jun#1{{\bf[#1 -- jun]}}
\def\bx{{\bf x}}
\def\br{{\bf r}}
\def\bv{{\bf v}}
\def\ba{{\bf a}}
\def\badot{{\bf \dot a}}
\def\batwo{{{\bf  a}^{(2)}}}
\def\bathree{{{\bf  a}^{(3)}}}
%
% ...........................................................................
%
\begin{document}

\begin{center}

{\lgggb Pure Gravity}

\bigskip

{\lggb Particles at Play}

\bigskip

\bigskip

\bigskip

{\slgh Piet Hut}

\bigskip

{\lr School of Natural Sciences}

\medskip

{\lr Institute for Advanced Study}

\medskip

{\lr Princeton, NJ 08540, U.S.A.}

\medskip

{{\it email:} piet@ias.edu}

\bigskip

\bigskip

{\slgh Jun Makino}

\bigskip

{\lr Department of Astronomy}

\medskip

{\lr University of Tokyo}

\medskip

{\lr 7-3-1 Hongo, Bunkyo-ku}

\medskip

{\lr Tokyo 113-0033, JAPAN}

\medskip

{{\it email:} makino@astron.s.u-tokyo.ac.jp}

\end{center}

\bigskip

\bigskip

\begin{abstract}

Chapter 1. [ {\it DRAFT -- DRAFT -- DRAFT -- DRAFT } ]

\end{abstract}

\newpage
\section{Introduction}

\subsection{2-body problem, forward-Euler}

2-body problem, first as 1-body problem of relative motion.

$$
\frac{d^2}{dt^2}\br = - G \frac{M_1 + M_2}{r^3}\br
$$

$G=1$

$M_1 + M_2 = 1$

\verbatiminput{forward_euler1.C}

\begin{verbatim}
g++ forward_euler1.C -o forward_euler1
\end{verbatim}

\begin{verbatim}
|gravity> forward_euler1 > for.out
|gravity> head for.out
1 0.005 0 -0.01 0.5 0
0.9999 0.01 0 -0.0199996 0.49995 0
0.9997 0.0149995 0 -0.0300001 0.49985 0
0.9994 0.019998 0 -0.0400027 0.4997 0
0.999 0.024995 0 -0.0500088 0.4995 0
0.9985 0.02999 0 -0.0600194 0.499249 0
0.9979 0.0349825 0 -0.0700359 0.498948 0
0.997199 0.039972 0 -0.0800595 0.498597 0
0.996399 0.0449579 0 -0.0900916 0.498195 0
0.995498 0.0499399 0 -0.100133 0.497742 0
|gravity> tail for.out
7.62061 -6.22601 0 0.812921 -0.574783 0
7.62874 -6.23176 0 0.812841 -0.574718 0
7.63687 -6.23751 0 0.812761 -0.574653 0
7.645 -6.24325 0 0.812682 -0.574588 0
7.65312 -6.249 0 0.812602 -0.574523 0
7.66125 -6.25475 0 0.812523 -0.574458 0
7.66937 -6.26049 0 0.812443 -0.574393 0
7.6775 -6.26623 0 0.812364 -0.574329 0
7.68562 -6.27198 0 0.812286 -0.574264 0
7.69375 -6.27772 0 0.812207 -0.5742 0
|gravity>
\end{verbatim}

ast: Unexpected large!! What happened.

comp: Perhaps a trap because you did not use ./forward\_euler1 ?

ast: baka, why?

comp: well, I'm not really serious about danger, but note that other
programs with the same name may be picked up.  For example, beginners
often call a program `test', which makes it quite likely that a
who-knows-where program will be executed instead.

ast: I doubt that there is a forward\_euler1 lurking somewhere in the
operating system.

comp: I think you're right -- now while we are writing this book.  But
notice that the first reader, after downloading the book, will indeed
have a program by that name somewhere on his or her system!  So I
still vote for using ./forward\_euler1.  Oh well, go ahead and live
dangerously if you want.  But don't tell me I didn't warn you!

ast: Let's look at the real danger here -- let's look at what went
wrong with the orbit.

\begin{verbatim}
|gravity> gnuplot
gnuplot> set size square
gnuplot> plot 'for.out'
\end{verbatim}

Note that the sentence ``set size square'' can be included in a file
called `.gnuplot'; we will not repeat this line here in further
examples of the use of gnuplot; we will leave it to the reader to add
this line to a `.gnuplot' file, or to add it manually at the beginning
of each example below.

\begin{figure}
\begin{center}
\leavevmode
\epsfxsize = 12 cm;
\epsffile{for.ps}
\caption{$dt = 0.01$}
\label{fig:for}
\end{center}
\end{figure}

ast: aha!

comp: aha??

ast: the time step must have been too large.

(With this program, the orbit flies apart around the first pericenter
passage, and the particles escape from each other.)

ast: by the way, I'd like to print out a copy.  How do we do that?

\begin{verbatim}
gnuplot> set terminal post eps 22
Terminal type set to 'postscript'
Options are 'eps noenhanced monochrome dashed defaultplex "Helvetica" 22'
gnuplot> set output "for.ps"
gnuplot> replot
gnuplot> 
\end{verbatim}

comp: now you can simply plot the figure by giving the command:

\begin{verbatim}
|gravity> lpr for.ps
|gravity>
\end{verbatim}

So let's make the time step size something we can specify, to make it
smaller:

\verbatiminput{forward_euler2.C}

First repeat the previous disaster.

For $dt = 0.01$, the same explosion happens.

\begin{verbatim}
g++ forward_euler2.C -o forward_euler2
|gravity> forward_euler2 > for1.out
1
Etot [t = 0] = -0.875
de = 6.24998e-05
de = 0.000124996
de = 0.000187497
de = 0.000250011
de = 0.000312547
. . . . .
de = 1.26899
de = 1.26899
de = 1.26899
|gravity> forward_euler2 > for10.out
10
Etot [t = 0] = -0.875
de = 6.25e-07
de = 6.87512e-06
de = 1.3126e-05
de = 1.93786e-05
. . . . .
de = 0.425318
de = 0.425319
de = 0.425319
|gravity> gnuplot
gnuplot> plot 'for10.out'
\end{verbatim}

For $dt = 0.001$, at least a few orbits are completed, but each orbit
quickly grows.

\begin{figure}
\begin{center}
\leavevmode
\epsfxsize = 12 cm
\epsffile{for10.ps}
\caption{$dt = 0.001$}
\label{fig:for10}
\end{center}
\end{figure}

For $dt = 0.0001$, the spiraling-out is calmer.

\begin{figure}
\begin{center}
\leavevmode
\epsfxsize = 12 cm
\epsffile{for100.ps}
\caption{$dt = 0.0001$}
\label{fig:for100}
\end{center}
\end{figure}

For $dt = 0.00001$, the spiraling-out is still barely visible under gnuplot

\begin{figure}
\begin{center}
\leavevmode
\epsfxsize = 12 cm
\epsffile{for1000.ps}
\caption{$dt = 0.00001$}
\label{fig:for1000}
\end{center}
\end{figure}

\subsection{2-body problem, leapfrog}

\verbatiminput{leapfrog1.C}

\begin{verbatim}
g++ leapfrog1.C -o leapfrog1
|gravity> leapfrog1 > leap1.out
1
Etot [t = 0] = -0.875
de = 1.17198e-09
de = 4.68985e-09
de = 1.05594e-08
de = 1.87905e-08
. . . . .
de = 3.6141e-05
de = 3.38728e-05
de = 3.17741e-05
|gravity> leapfrog1 > leap10.out
10
Etot [t = 0] = -0.875
de = 1.1724e-13
de = 1.41822e-11
de = 5.17113e-11
de = 1.12767e-10
. . . . .
de = 3.85679e-07
de = 3.61152e-07
de = 3.38488e-07
|gravity> leapfrog1 > leap100.out
100
Etot [t = 0] = -0.875
de = -1.11022e-16
de = 1.20126e-13
de = 4.74509e-13
de = 1.06393e-12
. . . . .
de = 3.88014e-09
de = 3.63308e-09
de = 3.40483e-09
|gravity> leapfrog1 > leap1000.out
1000
Etot [t = 0] = -0.875
de = 0
de = 5.55112e-16
de = 2.88658e-15
de = 9.10383e-15
. . . . .
de = 3.88177e-11
de = 3.6344e-11
de = 3.40558e-11
|gravity> gnuplot
gnuplot> plot 'leap1.out'
\end{verbatim}

\begin{figure}
\begin{center}
\leavevmode
\epsfxsize = 12 cm
\epsffile{leap1.ps}
\caption{$dt = 0.01$}
\label{fig:leap1}
\end{center}
\end{figure}

notice rotation of periastron: advance of apsidal nodes.
In other words, in the previous picture the apastron part of the orbit
was fattened; now the quadratures are the parts were the lines get thicker.

\begin{verbatim}
gnuplot> plot 'leap10.out'
\end{verbatim}

\begin{figure}
\begin{center}
\leavevmode
\epsfxsize = 12 cm
\epsffile{leap10.ps}
\caption{$dt = 0.001$}
\label{fig:leap10}
\end{center}
\end{figure}

Aha!  Now you cannot see the deviations anymore from orbit to orbit.

\subsection{3-body problem, leapfrog, circle}

Now we put three particles on a circle.  We use the balance of
centrifugal force $ \frac{v^2}{r} $ and the centripetal gravitational
force to enforce a circular orbit.

\verbatiminput{leapfrog1a.C}

\begin{verbatim}
g++ leapfrog1a.C -o leapfrog1a
|gravity> leapfrog1a > triple1a.out
1
Etot [t = 0] = -0.866025
de = 1.0103e-14
de = 4.16334e-14
de = 9.34808e-14
de = 1.66644e-13
. . . . .
de = 2.64304e-10
de = 2.66949e-10
de = 2.69598e-10
\end{verbatim}

\begin{figure}
\begin{center}
\leavevmode
\epsfxsize = 12 cm
\epsffile{triple1a.ps}
\caption{$dt = 0.01$}
\label{fig:triple1a}
\end{center}
\end{figure}

Now add a 0.001 perturbation to the velocity of particle 1, and run
for a longer time, 1000 instead of 10 time units:

\verbatiminput{leapfrog1b.C}

\begin{verbatim}
g++ leapfrog1b.C -o leapfrog1b
|gravity> leapfrog1b > triple1b.out
1
Etot [t = 0] = -0.866025
de = 8.33436e-11
de = 1.66702e-10
de = 2.50068e-10
de = 3.33439e-10
. . . . .
de = 1.48837e-08
de = 1.52172e-08
de = 1.5549e-08
\end{verbatim}

\begin{figure}
\begin{center}
\leavevmode
\epsfxsize = 12 cm
\epsffile{triple1b.ps}
\caption{$dt = 0.01$}
\label{fig:triple1b}
\end{center}
\end{figure}

The orbits begin to diverge.
Now twice as long:

\verbatiminput{leapfrog1c.C}

\begin{verbatim}
g++ leapfrog1c.C -o leapfrog1c
|gravity> leapfrog1c > triple1c.out
1
Etot [t = 0] = -0.866025
de = 8.33436e-11
de = 1.66702e-10
de = 2.50068e-10
de = 3.33439e-10
. . . . .
de = 8.25631e-06
de = 8.1391e-06
de = 8.02621e-06
\end{verbatim}

\begin{figure}
\begin{center}
\leavevmode
\epsfxsize = 12 cm
\epsffile{triple1c.ps}
\caption{$dt = 0.01$}
\label{fig:triple1c}
\end{center}
\end{figure}

The orbits begin to get chaotic.
Now four times as long:

\verbatiminput{leapfrog1d.C}

\begin{verbatim}
g++ leapfrog1d.C -o leapfrog1d
|gravity> leapfrog1d > triple1d.out
1
Etot [t = 0] = -0.866025
de = 8.33436e-11
de = 1.66702e-10
de = 2.50068e-10
de = 3.33439e-10
. . . . .
de = 29.1127
de = 29.1127
de = 29.1127
\end{verbatim}

Note that in gnuplot we want to give the extra commands:

\begin{verbatim}
set xrange [-5:5]
set yrange [-5:5]
\end{verbatim}

\begin{figure}
\begin{center}
\leavevmode
\epsfxsize = 12 cm
\epsffile{triple1d.ps}
\caption{$dt = 0.01$}
\label{fig:triple1d}
\end{center}
\end{figure}

Energy error causes explosion.
Now with ten times smaller time steps:

\begin{verbatim}
|gravity> leapfrog1d > triple1d1.out
10
Etot [t = 0] = -0.866025
de = 8.32667e-14
de = 9.17044e-13
de = 1.7506e-12
de = 2.58327e-12
. . . . .
de = 1.60324e-06
de = 1.91979e-06
de = 2.29792e-06
\end{verbatim}

\begin{figure}
\begin{center}
\leavevmode
\epsfxsize = 12 cm
\epsffile{triple1d1.ps}
\caption{$dt = 0.01$}
\label{fig:triple1d1}
\end{center}
\end{figure}

Great: decay of unstable triple.  But have we converged to reality?
Now with again ten times smaller time steps:

\begin{verbatim}
|gravity> leapfrog1d > triple1d2.out
100
Etot [t = 0] = -0.866025
de = 1.11022e-16
de = 7.66054e-15
de = 1.58762e-14
de = 2.39808e-14
. . . . .
de = 1.84725e-08
de = 2.07468e-08
de = 2.3423e-08
\end{verbatim}

\begin{figure}
\begin{center}
\leavevmode
\epsfxsize = 12 cm
\epsffile{triple1d2.ps}
\caption{$dt = 0.01$}
\label{fig:triple1d2}
\end{center}
\end{figure}

Great: We do seem to have converged to reality!
Now ten times longer:

\verbatiminput{leapfrog1e.C}

\begin{verbatim}
g++ leapfrog1e.C -o leapfrog1e
|gravity> leapfrog1e > triple1e.out
100
Etot [t = 0] = -0.866025
de = 1.11022e-16
de = 7.66054e-15
de = 1.58762e-14
de = 2.39808e-14
. . . . .
de = 1.88229e-09
de = 1.96058e-09
de = 2.04432e-09
\end{verbatim}

\begin{figure}
\begin{center}
\leavevmode
\epsfxsize = 12 cm
\epsffile{triple1e.ps}
\caption{$dt = 0.01$}
\label{fig:triple1e}
\end{center}
\end{figure}

By the way, notice how leapfrog recovers from $10^{-5}$ back to
$10^{-10}$ type errors.

\subsection{3-body problem, leapfrog, figure-8}

\verbatiminput{leapfrog2.C}

\begin{verbatim}
g++ leapfrog2.C -o leapfrog2
|gravity> leapfrog2 > triple1.out
1
Etot [t = 0] = -1.28705
de = -3.95102e-08
de = -1.58104e-07
de = -3.55968e-07
de = -6.33411e-07
. . . . .
de = -7.43225e-05
de = -7.50071e-05
de = -7.54851e-05
\end{verbatim}

\begin{figure}
\begin{center}
\leavevmode
\epsfxsize = 12 cm
\epsffile{triple1.ps}
\caption{$dt = 0.01$}
\label{fig:triple1}
\end{center}
\end{figure}

\subsection{modular leapfrog}

now a more modular version.

Note: extra header file for width of output field:

\begin{verbatim}
#include  <iomanip.h>
\end{verbatim}

\verbatiminput{leapfrog2.C}

But more compact, with the following input file:

\verbatiminput{triple.in}

we can use:

\verbatiminput{leapfrog3.C}

or slightly more compactly:

\verbatiminput{leapfrog4.C}

todo:

N = 2 --$>$ N

softening

refine --$>$ t\_out and t\_end

subroutines

initial conditions

better output



\bigskip

\bigskip

\bigskip

{\it Acknowledgements:}
We thank xxx, xxx and xxx for valuable
discussions.  This work is supported in part by the Research for the
Future Program of Japan Society for the Promotion of Science
(JSPS-RFTP97P01102).

\end{document}
